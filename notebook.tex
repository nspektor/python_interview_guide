
% Default to the notebook output style

    


% Inherit from the specified cell style.




    
\documentclass[11pt]{article}

    
    
    \usepackage[T1]{fontenc}
    % Nicer default font (+ math font) than Computer Modern for most use cases
    \usepackage{mathpazo}

    % Basic figure setup, for now with no caption control since it's done
    % automatically by Pandoc (which extracts ![](path) syntax from Markdown).
    \usepackage{graphicx}
    % We will generate all images so they have a width \maxwidth. This means
    % that they will get their normal width if they fit onto the page, but
    % are scaled down if they would overflow the margins.
    \makeatletter
    \def\maxwidth{\ifdim\Gin@nat@width>\linewidth\linewidth
    \else\Gin@nat@width\fi}
    \makeatother
    \let\Oldincludegraphics\includegraphics
    % Set max figure width to be 80% of text width, for now hardcoded.
    \renewcommand{\includegraphics}[1]{\Oldincludegraphics[width=.8\maxwidth]{#1}}
    % Ensure that by default, figures have no caption (until we provide a
    % proper Figure object with a Caption API and a way to capture that
    % in the conversion process - todo).
    \usepackage{caption}
    \DeclareCaptionLabelFormat{nolabel}{}
    \captionsetup{labelformat=nolabel}

    \usepackage{adjustbox} % Used to constrain images to a maximum size 
    \usepackage{xcolor} % Allow colors to be defined
    \usepackage{enumerate} % Needed for markdown enumerations to work
    \usepackage{geometry} % Used to adjust the document margins
    \usepackage{amsmath} % Equations
    \usepackage{amssymb} % Equations
    \usepackage{textcomp} % defines textquotesingle
    % Hack from http://tex.stackexchange.com/a/47451/13684:
    \AtBeginDocument{%
        \def\PYZsq{\textquotesingle}% Upright quotes in Pygmentized code
    }
    \usepackage{upquote} % Upright quotes for verbatim code
    \usepackage{eurosym} % defines \euro
    \usepackage[mathletters]{ucs} % Extended unicode (utf-8) support
    \usepackage[utf8x]{inputenc} % Allow utf-8 characters in the tex document
    \usepackage{fancyvrb} % verbatim replacement that allows latex
    \usepackage{grffile} % extends the file name processing of package graphics 
                         % to support a larger range 
    % The hyperref package gives us a pdf with properly built
    % internal navigation ('pdf bookmarks' for the table of contents,
    % internal cross-reference links, web links for URLs, etc.)
    \usepackage{hyperref}
    \usepackage{longtable} % longtable support required by pandoc >1.10
    \usepackage{booktabs}  % table support for pandoc > 1.12.2
    \usepackage[inline]{enumitem} % IRkernel/repr support (it uses the enumerate* environment)
    \usepackage[normalem]{ulem} % ulem is needed to support strikethroughs (\sout)
                                % normalem makes italics be italics, not underlines
    

    
    
    % Colors for the hyperref package
    \definecolor{urlcolor}{rgb}{0,.145,.698}
    \definecolor{linkcolor}{rgb}{.71,0.21,0.01}
    \definecolor{citecolor}{rgb}{.12,.54,.11}

    % ANSI colors
    \definecolor{ansi-black}{HTML}{3E424D}
    \definecolor{ansi-black-intense}{HTML}{282C36}
    \definecolor{ansi-red}{HTML}{E75C58}
    \definecolor{ansi-red-intense}{HTML}{B22B31}
    \definecolor{ansi-green}{HTML}{00A250}
    \definecolor{ansi-green-intense}{HTML}{007427}
    \definecolor{ansi-yellow}{HTML}{DDB62B}
    \definecolor{ansi-yellow-intense}{HTML}{B27D12}
    \definecolor{ansi-blue}{HTML}{208FFB}
    \definecolor{ansi-blue-intense}{HTML}{0065CA}
    \definecolor{ansi-magenta}{HTML}{D160C4}
    \definecolor{ansi-magenta-intense}{HTML}{A03196}
    \definecolor{ansi-cyan}{HTML}{60C6C8}
    \definecolor{ansi-cyan-intense}{HTML}{258F8F}
    \definecolor{ansi-white}{HTML}{C5C1B4}
    \definecolor{ansi-white-intense}{HTML}{A1A6B2}

    % commands and environments needed by pandoc snippets
    % extracted from the output of `pandoc -s`
    \providecommand{\tightlist}{%
      \setlength{\itemsep}{0pt}\setlength{\parskip}{0pt}}
    \DefineVerbatimEnvironment{Highlighting}{Verbatim}{commandchars=\\\{\}}
    % Add ',fontsize=\small' for more characters per line
    \newenvironment{Shaded}{}{}
    \newcommand{\KeywordTok}[1]{\textcolor[rgb]{0.00,0.44,0.13}{\textbf{{#1}}}}
    \newcommand{\DataTypeTok}[1]{\textcolor[rgb]{0.56,0.13,0.00}{{#1}}}
    \newcommand{\DecValTok}[1]{\textcolor[rgb]{0.25,0.63,0.44}{{#1}}}
    \newcommand{\BaseNTok}[1]{\textcolor[rgb]{0.25,0.63,0.44}{{#1}}}
    \newcommand{\FloatTok}[1]{\textcolor[rgb]{0.25,0.63,0.44}{{#1}}}
    \newcommand{\CharTok}[1]{\textcolor[rgb]{0.25,0.44,0.63}{{#1}}}
    \newcommand{\StringTok}[1]{\textcolor[rgb]{0.25,0.44,0.63}{{#1}}}
    \newcommand{\CommentTok}[1]{\textcolor[rgb]{0.38,0.63,0.69}{\textit{{#1}}}}
    \newcommand{\OtherTok}[1]{\textcolor[rgb]{0.00,0.44,0.13}{{#1}}}
    \newcommand{\AlertTok}[1]{\textcolor[rgb]{1.00,0.00,0.00}{\textbf{{#1}}}}
    \newcommand{\FunctionTok}[1]{\textcolor[rgb]{0.02,0.16,0.49}{{#1}}}
    \newcommand{\RegionMarkerTok}[1]{{#1}}
    \newcommand{\ErrorTok}[1]{\textcolor[rgb]{1.00,0.00,0.00}{\textbf{{#1}}}}
    \newcommand{\NormalTok}[1]{{#1}}
    
    % Additional commands for more recent versions of Pandoc
    \newcommand{\ConstantTok}[1]{\textcolor[rgb]{0.53,0.00,0.00}{{#1}}}
    \newcommand{\SpecialCharTok}[1]{\textcolor[rgb]{0.25,0.44,0.63}{{#1}}}
    \newcommand{\VerbatimStringTok}[1]{\textcolor[rgb]{0.25,0.44,0.63}{{#1}}}
    \newcommand{\SpecialStringTok}[1]{\textcolor[rgb]{0.73,0.40,0.53}{{#1}}}
    \newcommand{\ImportTok}[1]{{#1}}
    \newcommand{\DocumentationTok}[1]{\textcolor[rgb]{0.73,0.13,0.13}{\textit{{#1}}}}
    \newcommand{\AnnotationTok}[1]{\textcolor[rgb]{0.38,0.63,0.69}{\textbf{\textit{{#1}}}}}
    \newcommand{\CommentVarTok}[1]{\textcolor[rgb]{0.38,0.63,0.69}{\textbf{\textit{{#1}}}}}
    \newcommand{\VariableTok}[1]{\textcolor[rgb]{0.10,0.09,0.49}{{#1}}}
    \newcommand{\ControlFlowTok}[1]{\textcolor[rgb]{0.00,0.44,0.13}{\textbf{{#1}}}}
    \newcommand{\OperatorTok}[1]{\textcolor[rgb]{0.40,0.40,0.40}{{#1}}}
    \newcommand{\BuiltInTok}[1]{{#1}}
    \newcommand{\ExtensionTok}[1]{{#1}}
    \newcommand{\PreprocessorTok}[1]{\textcolor[rgb]{0.74,0.48,0.00}{{#1}}}
    \newcommand{\AttributeTok}[1]{\textcolor[rgb]{0.49,0.56,0.16}{{#1}}}
    \newcommand{\InformationTok}[1]{\textcolor[rgb]{0.38,0.63,0.69}{\textbf{\textit{{#1}}}}}
    \newcommand{\WarningTok}[1]{\textcolor[rgb]{0.38,0.63,0.69}{\textbf{\textit{{#1}}}}}
    
    
    % Define a nice break command that doesn't care if a line doesn't already
    % exist.
    \def\br{\hspace*{\fill} \\* }
    % Math Jax compatability definitions
    \def\gt{>}
    \def\lt{<}
    % Document parameters
    \title{Google Guide}
    
    
    

    % Pygments definitions
    
\makeatletter
\def\PY@reset{\let\PY@it=\relax \let\PY@bf=\relax%
    \let\PY@ul=\relax \let\PY@tc=\relax%
    \let\PY@bc=\relax \let\PY@ff=\relax}
\def\PY@tok#1{\csname PY@tok@#1\endcsname}
\def\PY@toks#1+{\ifx\relax#1\empty\else%
    \PY@tok{#1}\expandafter\PY@toks\fi}
\def\PY@do#1{\PY@bc{\PY@tc{\PY@ul{%
    \PY@it{\PY@bf{\PY@ff{#1}}}}}}}
\def\PY#1#2{\PY@reset\PY@toks#1+\relax+\PY@do{#2}}

\expandafter\def\csname PY@tok@w\endcsname{\def\PY@tc##1{\textcolor[rgb]{0.73,0.73,0.73}{##1}}}
\expandafter\def\csname PY@tok@c\endcsname{\let\PY@it=\textit\def\PY@tc##1{\textcolor[rgb]{0.25,0.50,0.50}{##1}}}
\expandafter\def\csname PY@tok@cp\endcsname{\def\PY@tc##1{\textcolor[rgb]{0.74,0.48,0.00}{##1}}}
\expandafter\def\csname PY@tok@k\endcsname{\let\PY@bf=\textbf\def\PY@tc##1{\textcolor[rgb]{0.00,0.50,0.00}{##1}}}
\expandafter\def\csname PY@tok@kp\endcsname{\def\PY@tc##1{\textcolor[rgb]{0.00,0.50,0.00}{##1}}}
\expandafter\def\csname PY@tok@kt\endcsname{\def\PY@tc##1{\textcolor[rgb]{0.69,0.00,0.25}{##1}}}
\expandafter\def\csname PY@tok@o\endcsname{\def\PY@tc##1{\textcolor[rgb]{0.40,0.40,0.40}{##1}}}
\expandafter\def\csname PY@tok@ow\endcsname{\let\PY@bf=\textbf\def\PY@tc##1{\textcolor[rgb]{0.67,0.13,1.00}{##1}}}
\expandafter\def\csname PY@tok@nb\endcsname{\def\PY@tc##1{\textcolor[rgb]{0.00,0.50,0.00}{##1}}}
\expandafter\def\csname PY@tok@nf\endcsname{\def\PY@tc##1{\textcolor[rgb]{0.00,0.00,1.00}{##1}}}
\expandafter\def\csname PY@tok@nc\endcsname{\let\PY@bf=\textbf\def\PY@tc##1{\textcolor[rgb]{0.00,0.00,1.00}{##1}}}
\expandafter\def\csname PY@tok@nn\endcsname{\let\PY@bf=\textbf\def\PY@tc##1{\textcolor[rgb]{0.00,0.00,1.00}{##1}}}
\expandafter\def\csname PY@tok@ne\endcsname{\let\PY@bf=\textbf\def\PY@tc##1{\textcolor[rgb]{0.82,0.25,0.23}{##1}}}
\expandafter\def\csname PY@tok@nv\endcsname{\def\PY@tc##1{\textcolor[rgb]{0.10,0.09,0.49}{##1}}}
\expandafter\def\csname PY@tok@no\endcsname{\def\PY@tc##1{\textcolor[rgb]{0.53,0.00,0.00}{##1}}}
\expandafter\def\csname PY@tok@nl\endcsname{\def\PY@tc##1{\textcolor[rgb]{0.63,0.63,0.00}{##1}}}
\expandafter\def\csname PY@tok@ni\endcsname{\let\PY@bf=\textbf\def\PY@tc##1{\textcolor[rgb]{0.60,0.60,0.60}{##1}}}
\expandafter\def\csname PY@tok@na\endcsname{\def\PY@tc##1{\textcolor[rgb]{0.49,0.56,0.16}{##1}}}
\expandafter\def\csname PY@tok@nt\endcsname{\let\PY@bf=\textbf\def\PY@tc##1{\textcolor[rgb]{0.00,0.50,0.00}{##1}}}
\expandafter\def\csname PY@tok@nd\endcsname{\def\PY@tc##1{\textcolor[rgb]{0.67,0.13,1.00}{##1}}}
\expandafter\def\csname PY@tok@s\endcsname{\def\PY@tc##1{\textcolor[rgb]{0.73,0.13,0.13}{##1}}}
\expandafter\def\csname PY@tok@sd\endcsname{\let\PY@it=\textit\def\PY@tc##1{\textcolor[rgb]{0.73,0.13,0.13}{##1}}}
\expandafter\def\csname PY@tok@si\endcsname{\let\PY@bf=\textbf\def\PY@tc##1{\textcolor[rgb]{0.73,0.40,0.53}{##1}}}
\expandafter\def\csname PY@tok@se\endcsname{\let\PY@bf=\textbf\def\PY@tc##1{\textcolor[rgb]{0.73,0.40,0.13}{##1}}}
\expandafter\def\csname PY@tok@sr\endcsname{\def\PY@tc##1{\textcolor[rgb]{0.73,0.40,0.53}{##1}}}
\expandafter\def\csname PY@tok@ss\endcsname{\def\PY@tc##1{\textcolor[rgb]{0.10,0.09,0.49}{##1}}}
\expandafter\def\csname PY@tok@sx\endcsname{\def\PY@tc##1{\textcolor[rgb]{0.00,0.50,0.00}{##1}}}
\expandafter\def\csname PY@tok@m\endcsname{\def\PY@tc##1{\textcolor[rgb]{0.40,0.40,0.40}{##1}}}
\expandafter\def\csname PY@tok@gh\endcsname{\let\PY@bf=\textbf\def\PY@tc##1{\textcolor[rgb]{0.00,0.00,0.50}{##1}}}
\expandafter\def\csname PY@tok@gu\endcsname{\let\PY@bf=\textbf\def\PY@tc##1{\textcolor[rgb]{0.50,0.00,0.50}{##1}}}
\expandafter\def\csname PY@tok@gd\endcsname{\def\PY@tc##1{\textcolor[rgb]{0.63,0.00,0.00}{##1}}}
\expandafter\def\csname PY@tok@gi\endcsname{\def\PY@tc##1{\textcolor[rgb]{0.00,0.63,0.00}{##1}}}
\expandafter\def\csname PY@tok@gr\endcsname{\def\PY@tc##1{\textcolor[rgb]{1.00,0.00,0.00}{##1}}}
\expandafter\def\csname PY@tok@ge\endcsname{\let\PY@it=\textit}
\expandafter\def\csname PY@tok@gs\endcsname{\let\PY@bf=\textbf}
\expandafter\def\csname PY@tok@gp\endcsname{\let\PY@bf=\textbf\def\PY@tc##1{\textcolor[rgb]{0.00,0.00,0.50}{##1}}}
\expandafter\def\csname PY@tok@go\endcsname{\def\PY@tc##1{\textcolor[rgb]{0.53,0.53,0.53}{##1}}}
\expandafter\def\csname PY@tok@gt\endcsname{\def\PY@tc##1{\textcolor[rgb]{0.00,0.27,0.87}{##1}}}
\expandafter\def\csname PY@tok@err\endcsname{\def\PY@bc##1{\setlength{\fboxsep}{0pt}\fcolorbox[rgb]{1.00,0.00,0.00}{1,1,1}{\strut ##1}}}
\expandafter\def\csname PY@tok@kc\endcsname{\let\PY@bf=\textbf\def\PY@tc##1{\textcolor[rgb]{0.00,0.50,0.00}{##1}}}
\expandafter\def\csname PY@tok@kd\endcsname{\let\PY@bf=\textbf\def\PY@tc##1{\textcolor[rgb]{0.00,0.50,0.00}{##1}}}
\expandafter\def\csname PY@tok@kn\endcsname{\let\PY@bf=\textbf\def\PY@tc##1{\textcolor[rgb]{0.00,0.50,0.00}{##1}}}
\expandafter\def\csname PY@tok@kr\endcsname{\let\PY@bf=\textbf\def\PY@tc##1{\textcolor[rgb]{0.00,0.50,0.00}{##1}}}
\expandafter\def\csname PY@tok@bp\endcsname{\def\PY@tc##1{\textcolor[rgb]{0.00,0.50,0.00}{##1}}}
\expandafter\def\csname PY@tok@fm\endcsname{\def\PY@tc##1{\textcolor[rgb]{0.00,0.00,1.00}{##1}}}
\expandafter\def\csname PY@tok@vc\endcsname{\def\PY@tc##1{\textcolor[rgb]{0.10,0.09,0.49}{##1}}}
\expandafter\def\csname PY@tok@vg\endcsname{\def\PY@tc##1{\textcolor[rgb]{0.10,0.09,0.49}{##1}}}
\expandafter\def\csname PY@tok@vi\endcsname{\def\PY@tc##1{\textcolor[rgb]{0.10,0.09,0.49}{##1}}}
\expandafter\def\csname PY@tok@vm\endcsname{\def\PY@tc##1{\textcolor[rgb]{0.10,0.09,0.49}{##1}}}
\expandafter\def\csname PY@tok@sa\endcsname{\def\PY@tc##1{\textcolor[rgb]{0.73,0.13,0.13}{##1}}}
\expandafter\def\csname PY@tok@sb\endcsname{\def\PY@tc##1{\textcolor[rgb]{0.73,0.13,0.13}{##1}}}
\expandafter\def\csname PY@tok@sc\endcsname{\def\PY@tc##1{\textcolor[rgb]{0.73,0.13,0.13}{##1}}}
\expandafter\def\csname PY@tok@dl\endcsname{\def\PY@tc##1{\textcolor[rgb]{0.73,0.13,0.13}{##1}}}
\expandafter\def\csname PY@tok@s2\endcsname{\def\PY@tc##1{\textcolor[rgb]{0.73,0.13,0.13}{##1}}}
\expandafter\def\csname PY@tok@sh\endcsname{\def\PY@tc##1{\textcolor[rgb]{0.73,0.13,0.13}{##1}}}
\expandafter\def\csname PY@tok@s1\endcsname{\def\PY@tc##1{\textcolor[rgb]{0.73,0.13,0.13}{##1}}}
\expandafter\def\csname PY@tok@mb\endcsname{\def\PY@tc##1{\textcolor[rgb]{0.40,0.40,0.40}{##1}}}
\expandafter\def\csname PY@tok@mf\endcsname{\def\PY@tc##1{\textcolor[rgb]{0.40,0.40,0.40}{##1}}}
\expandafter\def\csname PY@tok@mh\endcsname{\def\PY@tc##1{\textcolor[rgb]{0.40,0.40,0.40}{##1}}}
\expandafter\def\csname PY@tok@mi\endcsname{\def\PY@tc##1{\textcolor[rgb]{0.40,0.40,0.40}{##1}}}
\expandafter\def\csname PY@tok@il\endcsname{\def\PY@tc##1{\textcolor[rgb]{0.40,0.40,0.40}{##1}}}
\expandafter\def\csname PY@tok@mo\endcsname{\def\PY@tc##1{\textcolor[rgb]{0.40,0.40,0.40}{##1}}}
\expandafter\def\csname PY@tok@ch\endcsname{\let\PY@it=\textit\def\PY@tc##1{\textcolor[rgb]{0.25,0.50,0.50}{##1}}}
\expandafter\def\csname PY@tok@cm\endcsname{\let\PY@it=\textit\def\PY@tc##1{\textcolor[rgb]{0.25,0.50,0.50}{##1}}}
\expandafter\def\csname PY@tok@cpf\endcsname{\let\PY@it=\textit\def\PY@tc##1{\textcolor[rgb]{0.25,0.50,0.50}{##1}}}
\expandafter\def\csname PY@tok@c1\endcsname{\let\PY@it=\textit\def\PY@tc##1{\textcolor[rgb]{0.25,0.50,0.50}{##1}}}
\expandafter\def\csname PY@tok@cs\endcsname{\let\PY@it=\textit\def\PY@tc##1{\textcolor[rgb]{0.25,0.50,0.50}{##1}}}

\def\PYZbs{\char`\\}
\def\PYZus{\char`\_}
\def\PYZob{\char`\{}
\def\PYZcb{\char`\}}
\def\PYZca{\char`\^}
\def\PYZam{\char`\&}
\def\PYZlt{\char`\<}
\def\PYZgt{\char`\>}
\def\PYZsh{\char`\#}
\def\PYZpc{\char`\%}
\def\PYZdl{\char`\$}
\def\PYZhy{\char`\-}
\def\PYZsq{\char`\'}
\def\PYZdq{\char`\"}
\def\PYZti{\char`\~}
% for compatibility with earlier versions
\def\PYZat{@}
\def\PYZlb{[}
\def\PYZrb{]}
\makeatother


    % Exact colors from NB
    \definecolor{incolor}{rgb}{0.0, 0.0, 0.5}
    \definecolor{outcolor}{rgb}{0.545, 0.0, 0.0}



    
    % Prevent overflowing lines due to hard-to-break entities
    \sloppy 
    % Setup hyperref package
    \hypersetup{
      breaklinks=true,  % so long urls are correctly broken across lines
      colorlinks=true,
      urlcolor=urlcolor,
      linkcolor=linkcolor,
      citecolor=citecolor,
      }
    % Slightly bigger margins than the latex defaults
    
    \geometry{verbose,tmargin=1in,bmargin=1in,lmargin=1in,rmargin=1in}
    
    

    \begin{document}
    
    
    \maketitle
    
    

    
    \section{Mark Abramov's super awesome Google interview
guide}\label{mark-abramovs-super-awesome-google-interview-guide}

\subsection{Made by your 1 tru 😍}\label{made-by-your-1-tru}

\section{Table of Contents}\label{table-of-contents}

\begin{enumerate}
\def\labelenumi{\arabic{enumi}.}
\tightlist
\item
  Data Structures

  \begin{itemize}
  \tightlist
  \item
    Lists
  \item
    Dictionaries
  \item
    Sets
  \item
    Binary Trees
  \end{itemize}
\item
  Graphs
\item
  BFS
\item
  DFS
\item
  Objects
\end{enumerate}

    \subsection{Data Structures}\label{data-structures}

\subsubsection{Lists}\label{lists}

    \begin{Verbatim}[commandchars=\\\{\}]
{\color{incolor}In [{\color{incolor}1}]:} \PY{n}{greeting} \PY{o}{=} \PY{n+nb}{list}\PY{p}{(}\PY{l+s+s1}{\PYZsq{}}\PY{l+s+s1}{HiMark}\PY{l+s+s1}{\PYZsq{}}\PY{p}{)} \PY{c+c1}{\PYZsh{}string to list}
        \PY{n}{greeting}
\end{Verbatim}


\begin{Verbatim}[commandchars=\\\{\}]
{\color{outcolor}Out[{\color{outcolor}1}]:} ['H', 'i', 'M', 'a', 'r', 'k']
\end{Verbatim}
            
    \begin{Verbatim}[commandchars=\\\{\}]
{\color{incolor}In [{\color{incolor}2}]:} \PY{n}{greeting}\PY{o}{.}\PY{n}{append}\PY{p}{(}\PY{l+s+s1}{\PYZsq{}}\PY{l+s+s1}{!}\PY{l+s+s1}{\PYZsq{}}\PY{p}{)} \PY{c+c1}{\PYZsh{}adds 1 element to the end}
\end{Verbatim}


    \begin{Verbatim}[commandchars=\\\{\}]
{\color{incolor}In [{\color{incolor}3}]:} \PY{n}{greeting}\PY{o}{.}\PY{n}{extend}\PY{p}{(}\PY{p}{[}\PY{l+s+s1}{\PYZsq{}}\PY{l+s+s1}{\PYZlt{}}\PY{l+s+s1}{\PYZsq{}}\PY{p}{,} \PY{l+s+s1}{\PYZsq{}}\PY{l+s+s1}{3}\PY{l+s+s1}{\PYZsq{}}\PY{p}{]}\PY{p}{)} \PY{c+c1}{\PYZsh{}extend appends all the elements of the given list}
\end{Verbatim}


    \begin{Verbatim}[commandchars=\\\{\}]
{\color{incolor}In [{\color{incolor}4}]:} \PY{n}{greeting}
\end{Verbatim}


\begin{Verbatim}[commandchars=\\\{\}]
{\color{outcolor}Out[{\color{outcolor}4}]:} ['H', 'i', 'M', 'a', 'r', 'k', '!', '<', '3']
\end{Verbatim}
            
    \begin{Verbatim}[commandchars=\\\{\}]
{\color{incolor}In [{\color{incolor}5}]:} \PY{n}{greet\PYZus{}string} \PY{o}{=} \PY{l+s+s1}{\PYZsq{}}\PY{l+s+s1}{\PYZsq{}}\PY{o}{.}\PY{n}{join}\PY{p}{(}\PY{n}{greeting}\PY{p}{)} \PY{c+c1}{\PYZsh{}joins with \PYZsq{}\PYZsq{} between every element}
\end{Verbatim}


    \begin{Verbatim}[commandchars=\\\{\}]
{\color{incolor}In [{\color{incolor}6}]:} \PY{n}{greeting} \PY{o}{*} \PY{l+m+mi}{2} \PY{c+c1}{\PYZsh{} the list twice}
\end{Verbatim}


\begin{Verbatim}[commandchars=\\\{\}]
{\color{outcolor}Out[{\color{outcolor}6}]:} ['H',
         'i',
         'M',
         'a',
         'r',
         'k',
         '!',
         '<',
         '3',
         'H',
         'i',
         'M',
         'a',
         'r',
         'k',
         '!',
         '<',
         '3']
\end{Verbatim}
            
    \begin{Verbatim}[commandchars=\\\{\}]
{\color{incolor}In [{\color{incolor}7}]:} \PY{n}{greet\PYZus{}string} \PY{o}{*} \PY{l+m+mi}{2} \PY{c+c1}{\PYZsh{}the string twice}
\end{Verbatim}


\begin{Verbatim}[commandchars=\\\{\}]
{\color{outcolor}Out[{\color{outcolor}7}]:} 'HiMark!<3HiMark!<3'
\end{Verbatim}
            
    Converting to Ascii: - ord(c) converts a character to its ascii value -
doing this in a list comprehenstion (thats what the {[} for in {]} thing
is called) does it for every char in the string

    \begin{Verbatim}[commandchars=\\\{\}]
{\color{incolor}In [{\color{incolor}8}]:} \PY{n}{greet\PYZus{}ascii} \PY{o}{=} \PY{p}{[}\PY{n+nb}{ord}\PY{p}{(}\PY{n}{c}\PY{p}{)} \PY{k}{for} \PY{n}{c} \PY{o+ow}{in} \PY{n}{greet\PYZus{}string}\PY{p}{]}
        \PY{n}{greet\PYZus{}ascii}
\end{Verbatim}


\begin{Verbatim}[commandchars=\\\{\}]
{\color{outcolor}Out[{\color{outcolor}8}]:} [72, 105, 77, 97, 114, 107, 33, 60, 51]
\end{Verbatim}
            
    \begin{Verbatim}[commandchars=\\\{\}]
{\color{incolor}In [{\color{incolor}9}]:} \PY{l+s+s1}{\PYZsq{}}\PY{l+s+s1}{\PYZsq{}}\PY{o}{.}\PY{n}{join}\PY{p}{(}\PY{n+nb}{chr}\PY{p}{(}\PY{n}{i}\PY{p}{)} \PY{k}{for} \PY{n}{i} \PY{o+ow}{in} \PY{n}{greet\PYZus{}ascii}\PY{p}{)} \PY{c+c1}{\PYZsh{} you can do a little list comprehension inside a method call without the []}
\end{Verbatim}


\begin{Verbatim}[commandchars=\\\{\}]
{\color{outcolor}Out[{\color{outcolor}9}]:} 'HiMark!<3'
\end{Verbatim}
            
    \begin{Verbatim}[commandchars=\\\{\}]
{\color{incolor}In [{\color{incolor}10}]:} \PY{n}{sorted\PYZus{}ascii} \PY{o}{=} \PY{n+nb}{sorted}\PY{p}{(}\PY{n}{greet\PYZus{}ascii}\PY{p}{)}
         \PY{n}{sorted\PYZus{}ascii}
\end{Verbatim}


\begin{Verbatim}[commandchars=\\\{\}]
{\color{outcolor}Out[{\color{outcolor}10}]:} [33, 51, 60, 72, 77, 97, 105, 107, 114]
\end{Verbatim}
            
    \begin{itemize}
\tightlist
\item
  you can use the optional key argument and pass it a lambda function
\item
  \texttt{lambda\ VARIABLE\_NAME\_YOU\_DEFINE:\ whatever\ you\ want\ to\ do\ to\ the\ VAR}
\end{itemize}

    \begin{Verbatim}[commandchars=\\\{\}]
{\color{incolor}In [{\color{incolor}11}]:} \PY{n}{greeting}\PY{o}{.}\PY{n}{sort}\PY{p}{(}\PY{n}{key} \PY{o}{=} \PY{k}{lambda} \PY{n}{char}\PY{p}{:} \PY{n}{char}\PY{o}{.}\PY{n}{upper}\PY{p}{(}\PY{p}{)}\PY{p}{)} 
         \PY{n}{greeting} \PY{o}{=} \PY{n}{greeting}\PY{p}{[}\PY{p}{:}\PY{p}{:}\PY{o}{\PYZhy{}}\PY{l+m+mi}{1}\PY{p}{]} \PY{c+c1}{\PYZsh{} REVERSE the list}
         \PY{n}{greeting}
\end{Verbatim}


\begin{Verbatim}[commandchars=\\\{\}]
{\color{outcolor}Out[{\color{outcolor}11}]:} ['r', 'M', 'k', 'i', 'H', 'a', '<', '3', '!']
\end{Verbatim}
            
    \paragraph{List Complexities}\label{list-complexities}

\begin{longtable}[]{@{}ll@{}}
\toprule
Operation & Big O\tabularnewline
\midrule
\endhead
append & O(1)\tabularnewline
Pop Last & O(1)\tabularnewline
Insert & O(n)\tabularnewline
Get/Set & O(1)\tabularnewline
Length & O(1)\tabularnewline
Sort & O(n log n)\tabularnewline
\bottomrule
\end{longtable}

    \begin{center}\rule{0.5\linewidth}{\linethickness}\end{center}

\subsubsection{Dictionaries}\label{dictionaries}

\begin{itemize}
\tightlist
\item
  \texttt{len(d)} returns the number of key,value pairs
\item
  \texttt{del\ d{[}k{]}} deletes the key k and its value
\item
  \texttt{d.pop(k)} deletes key and value and returns the value

  \begin{itemize}
  \tightlist
  \item
    can put a default value for if k isnt in the dict:
    \texttt{d.pop(k,\ \textquotesingle{}nope\textquotesingle{})}
  \item
    otherwise there will be a \emph{KeyError}
  \end{itemize}
\item
  \texttt{if\ k\ in\ d} if there is a key k in dictionary d
\item
  \texttt{d1.update(d2)} merges the dictionaries
\item
  \texttt{L\ =\ list(D)} converts dictionary into list of 2-item tuples
\item
  \texttt{D\ =\ dict(zip(L1,\ L2))} zips and converts 2 lists into a
  dictionary
\end{itemize}

    \begin{Verbatim}[commandchars=\\\{\}]
{\color{incolor}In [{\color{incolor}12}]:} \PY{n}{rappers} \PY{o}{=} \PY{p}{\PYZob{}}\PY{p}{\PYZcb{}} \PY{c+c1}{\PYZsh{} creating empty dictionary}
\end{Verbatim}


    \begin{Verbatim}[commandchars=\\\{\}]
{\color{incolor}In [{\color{incolor}13}]:} \PY{n}{rappers}\PY{p}{[}\PY{l+s+s1}{\PYZsq{}}\PY{l+s+s1}{Kanye}\PY{l+s+s1}{\PYZsq{}}\PY{p}{]} \PY{o}{=} \PY{l+m+mi}{1}
         \PY{n}{rappers}\PY{p}{[}\PY{l+s+s1}{\PYZsq{}}\PY{l+s+s1}{J Cole}\PY{l+s+s1}{\PYZsq{}}\PY{p}{]} \PY{o}{=} \PY{l+m+mi}{2}
         \PY{n}{rappers}\PY{p}{[}\PY{l+s+s1}{\PYZsq{}}\PY{l+s+s1}{Nicki Minaj}\PY{l+s+s1}{\PYZsq{}}\PY{p}{]} \PY{o}{=} \PY{l+m+mi}{15}
         \PY{n}{rappers}
\end{Verbatim}


\begin{Verbatim}[commandchars=\\\{\}]
{\color{outcolor}Out[{\color{outcolor}13}]:} \{'Kanye': 1, 'J Cole': 2, 'Nicki Minaj': 15\}
\end{Verbatim}
            
    \begin{Verbatim}[commandchars=\\\{\}]
{\color{incolor}In [{\color{incolor}14}]:} \PY{n+nb}{len}\PY{p}{(}\PY{n}{rappers}\PY{p}{)}
\end{Verbatim}


\begin{Verbatim}[commandchars=\\\{\}]
{\color{outcolor}Out[{\color{outcolor}14}]:} 3
\end{Verbatim}
            
    \begin{Verbatim}[commandchars=\\\{\}]
{\color{incolor}In [{\color{incolor}15}]:} \PY{k}{for} \PY{n}{ranking} \PY{o+ow}{in} \PY{n}{rappers}\PY{o}{.}\PY{n}{values}\PY{p}{(}\PY{p}{)}\PY{p}{:} \PY{c+c1}{\PYZsh{}also: rappers.keys()}
             \PY{n+nb}{print}\PY{p}{(}\PY{n}{ranking}\PY{p}{)}
\end{Verbatim}


    \begin{Verbatim}[commandchars=\\\{\}]
1
2
15

    \end{Verbatim}

    \begin{Verbatim}[commandchars=\\\{\}]
{\color{incolor}In [{\color{incolor}16}]:} \PY{k}{for} \PY{n}{rapper}\PY{p}{,} \PY{n}{ranking} \PY{o+ow}{in} \PY{n}{rappers}\PY{o}{.}\PY{n}{items}\PY{p}{(}\PY{p}{)}\PY{p}{:}
             \PY{n+nb}{print}\PY{p}{(}\PY{l+s+s1}{\PYZsq{}}\PY{l+s+s1}{the number }\PY{l+s+s1}{\PYZsq{}}\PY{p}{,} \PY{n}{ranking}\PY{p}{,} \PY{l+s+s1}{\PYZsq{}}\PY{l+s+s1}{ rapper is }\PY{l+s+s1}{\PYZsq{}}\PY{p}{,} \PY{n}{rapper}\PY{p}{)}
\end{Verbatim}


    \begin{Verbatim}[commandchars=\\\{\}]
the number  1  rapper is  Kanye
the number  2  rapper is  J Cole
the number  15  rapper is  Nicki Minaj

    \end{Verbatim}

    \paragraph{Dictionary Complexities}\label{dictionary-complexities}

\begin{longtable}[]{@{}ll@{}}
\toprule
Operation & Big O\tabularnewline
\midrule
\endhead
Delete & O(1)\tabularnewline
Insert & O(n)\tabularnewline
Get/Set & O(1)\tabularnewline
Length & O(1)\tabularnewline
Sort & O(n log n)\tabularnewline
\bottomrule
\end{longtable}

    \begin{center}\rule{0.5\linewidth}{\linethickness}\end{center}

\subsubsection{Sets}\label{sets}

\begin{itemize}
\tightlist
\item
  like a 1 dimensional dictionary
\item
  unordered, no duplicates
\end{itemize}

    \begin{Verbatim}[commandchars=\\\{\}]
{\color{incolor}In [{\color{incolor}17}]:} \PY{n}{your\PYZus{}friends} \PY{o}{=} \PY{n+nb}{set}\PY{p}{(}\PY{p}{)}
         \PY{n}{your\PYZus{}friends}\PY{o}{.}\PY{n}{add}\PY{p}{(}\PY{l+s+s2}{\PYZdq{}}\PY{l+s+s2}{Maxwell}\PY{l+s+s2}{\PYZdq{}}\PY{p}{)}
         \PY{n}{your\PYZus{}friends}\PY{o}{.}\PY{n}{add}\PY{p}{(}\PY{l+s+s2}{\PYZdq{}}\PY{l+s+s2}{Stelios}\PY{l+s+s2}{\PYZdq{}}\PY{p}{)}
         \PY{n}{your\PYZus{}friends}\PY{o}{.}\PY{n}{add}\PY{p}{(}\PY{l+s+s2}{\PYZdq{}}\PY{l+s+s2}{Jazmyn}\PY{l+s+s2}{\PYZdq{}}\PY{p}{)}
         \PY{n}{your\PYZus{}friends}
\end{Verbatim}


\begin{Verbatim}[commandchars=\\\{\}]
{\color{outcolor}Out[{\color{outcolor}17}]:} \{'Jazmyn', 'Maxwell', 'Stelios'\}
\end{Verbatim}
            
    \begin{Verbatim}[commandchars=\\\{\}]
{\color{incolor}In [{\color{incolor}18}]:} \PY{n}{your\PYZus{}friends}\PY{o}{.}\PY{n}{add}\PY{p}{(}\PY{l+s+s2}{\PYZdq{}}\PY{l+s+s2}{Maxwell}\PY{l+s+s2}{\PYZdq{}}\PY{p}{)}
         \PY{n}{your\PYZus{}friends}
\end{Verbatim}


\begin{Verbatim}[commandchars=\\\{\}]
{\color{outcolor}Out[{\color{outcolor}18}]:} \{'Jazmyn', 'Maxwell', 'Stelios'\}
\end{Verbatim}
            
    \begin{verbatim}
- oh well, guess you cant have 2 Maxwells
\end{verbatim}

    \paragraph{Set Complexities}\label{set-complexities}

\begin{longtable}[]{@{}ll@{}}
\toprule
Operation & Big O\tabularnewline
\midrule
\endhead
Delete & O(1)\tabularnewline
x in s & O(1)\tabularnewline
Get/Set & O(1)\tabularnewline
Length & O(1)\tabularnewline
\bottomrule
\end{longtable}

    \subsubsection{Binary Trees}\label{binary-trees}

\begin{itemize}
\tightlist
\item
  not native to python but heres a nice implementation (and a little
  intro to how classes work)
\end{itemize}

    \begin{Verbatim}[commandchars=\\\{\}]
{\color{incolor}In [{\color{incolor}19}]:} \PY{k}{class} \PY{n+nc}{Node}\PY{p}{:}
         
             \PY{k}{def} \PY{n+nf}{\PYZus{}\PYZus{}init\PYZus{}\PYZus{}}\PY{p}{(}\PY{n+nb+bp}{self}\PY{p}{,} \PY{n}{data}\PY{p}{)}\PY{p}{:}
                 \PY{c+c1}{\PYZsh{} init is the constructor}
                 \PY{n+nb+bp}{self}\PY{o}{.}\PY{n}{left} \PY{o}{=} \PY{k+kc}{None}
                 \PY{n+nb+bp}{self}\PY{o}{.}\PY{n}{right} \PY{o}{=} \PY{k+kc}{None}
                 \PY{n+nb+bp}{self}\PY{o}{.}\PY{n}{data} \PY{o}{=} \PY{n}{data}
         
             \PY{k}{def} \PY{n+nf}{insert}\PY{p}{(}\PY{n+nb+bp}{self}\PY{p}{,} \PY{n}{data}\PY{p}{)}\PY{p}{:} \PY{c+c1}{\PYZsh{} every method takes self}
                 \PY{c+c1}{\PYZsh{} Compare the new value with the parent node}
                 \PY{k}{if} \PY{n+nb+bp}{self}\PY{o}{.}\PY{n}{data}\PY{p}{:}
                     \PY{k}{if} \PY{n}{data} \PY{o}{\PYZlt{}} \PY{n+nb+bp}{self}\PY{o}{.}\PY{n}{data}\PY{p}{:}
                         \PY{k}{if} \PY{n+nb+bp}{self}\PY{o}{.}\PY{n}{left} \PY{o+ow}{is} \PY{k+kc}{None}\PY{p}{:}
                             \PY{n+nb+bp}{self}\PY{o}{.}\PY{n}{left} \PY{o}{=} \PY{n}{Node}\PY{p}{(}\PY{n}{data}\PY{p}{)}
                         \PY{k}{else}\PY{p}{:}
                             \PY{n+nb+bp}{self}\PY{o}{.}\PY{n}{left}\PY{o}{.}\PY{n}{insert}\PY{p}{(}\PY{n}{data}\PY{p}{)}
                     \PY{k}{elif} \PY{n}{data} \PY{o}{\PYZgt{}} \PY{n+nb+bp}{self}\PY{o}{.}\PY{n}{data}\PY{p}{:}
                         \PY{k}{if} \PY{n+nb+bp}{self}\PY{o}{.}\PY{n}{right} \PY{o+ow}{is} \PY{k+kc}{None}\PY{p}{:}
                             \PY{n+nb+bp}{self}\PY{o}{.}\PY{n}{right} \PY{o}{=} \PY{n}{Node}\PY{p}{(}\PY{n}{data}\PY{p}{)}
                         \PY{k}{else}\PY{p}{:}
                             \PY{n+nb+bp}{self}\PY{o}{.}\PY{n}{right}\PY{o}{.}\PY{n}{insert}\PY{p}{(}\PY{n}{data}\PY{p}{)}
                 \PY{k}{else}\PY{p}{:}
                     \PY{n+nb+bp}{self}\PY{o}{.}\PY{n}{data} \PY{o}{=} \PY{n}{data}
         
         \PY{c+c1}{\PYZsh{} Print the tree \PYZhy{} not a great representation but ya know}
             \PY{k}{def} \PY{n+nf}{PrintTree}\PY{p}{(}\PY{n+nb+bp}{self}\PY{p}{)}\PY{p}{:}
                 \PY{k}{if} \PY{n+nb+bp}{self}\PY{o}{.}\PY{n}{left}\PY{p}{:}
                     \PY{n+nb+bp}{self}\PY{o}{.}\PY{n}{left}\PY{o}{.}\PY{n}{PrintTree}\PY{p}{(}\PY{p}{)}
                 \PY{n+nb}{print}\PY{p}{(}\PY{n+nb+bp}{self}\PY{o}{.}\PY{n}{data}\PY{p}{)}\PY{p}{,}
                 \PY{k}{if} \PY{n+nb+bp}{self}\PY{o}{.}\PY{n}{right}\PY{p}{:}
                     \PY{n+nb+bp}{self}\PY{o}{.}\PY{n}{right}\PY{o}{.}\PY{n}{PrintTree}\PY{p}{(}\PY{p}{)}
         
         \PY{c+c1}{\PYZsh{} Use the insert method to add nodes}
         \PY{n}{root} \PY{o}{=} \PY{n}{Node}\PY{p}{(}\PY{l+m+mi}{12}\PY{p}{)}
         \PY{n}{root}\PY{o}{.}\PY{n}{insert}\PY{p}{(}\PY{l+m+mi}{6}\PY{p}{)}
         \PY{n}{root}\PY{o}{.}\PY{n}{insert}\PY{p}{(}\PY{l+m+mi}{14}\PY{p}{)}
         \PY{n}{root}\PY{o}{.}\PY{n}{insert}\PY{p}{(}\PY{l+m+mi}{3}\PY{p}{)}
         
         \PY{n}{root}\PY{o}{.}\PY{n}{PrintTree}\PY{p}{(}\PY{p}{)}\PY{c+c1}{\PYZsh{} notice you dont have to write self}
\end{Verbatim}


    \begin{Verbatim}[commandchars=\\\{\}]
3
6
12
14

    \end{Verbatim}

    \paragraph{Binary Tree Complexities}\label{binary-tree-complexities}

\begin{longtable}[]{@{}ll@{}}
\toprule
Operation & Big O\tabularnewline
\midrule
\endhead
Search & O(n)\tabularnewline
Insertion & O(n)\tabularnewline
Deletion & O(n)\tabularnewline
\bottomrule
\end{longtable}

\paragraph{Binary Search Tree
Complexities}\label{binary-search-tree-complexities}

\begin{longtable}[]{@{}ll@{}}
\toprule
Operation & Big O\tabularnewline
\midrule
\endhead
Search & O(h)-height\tabularnewline
Insertion & O(h)\tabularnewline
Deletion & O(h)\tabularnewline
\bottomrule
\end{longtable}

\paragraph{AVL Tree Complexities}\label{avl-tree-complexities}

\begin{longtable}[]{@{}ll@{}}
\toprule
Operation & Big O\tabularnewline
\midrule
\endhead
Search & O(log n)\tabularnewline
Insertion & O(log n)\tabularnewline
Deletion & O(log n)\tabularnewline
\bottomrule
\end{longtable}

    \subsubsection{Graphs}\label{graphs}

\begin{itemize}
\tightlist
\item
  can be represented in Python as a dictionary
  \includegraphics{http://faculty.cs.niu.edu/~freedman/340/340notes/gifImages/340graph7.gif}
\end{itemize}

    \begin{Verbatim}[commandchars=\\\{\}]
{\color{incolor}In [{\color{incolor}20}]:} \PY{n}{graph} \PY{o}{=} \PY{p}{\PYZob{}}
             \PY{l+s+s1}{\PYZsq{}}\PY{l+s+s1}{v1}\PY{l+s+s1}{\PYZsq{}}\PY{p}{:} \PY{p}{[}\PY{l+s+s1}{\PYZsq{}}\PY{l+s+s1}{v2}\PY{l+s+s1}{\PYZsq{}}\PY{p}{,} \PY{l+s+s1}{\PYZsq{}}\PY{l+s+s1}{v4}\PY{l+s+s1}{\PYZsq{}}\PY{p}{,} \PY{l+s+s1}{\PYZsq{}}\PY{l+s+s1}{v3}\PY{l+s+s1}{\PYZsq{}}\PY{p}{]}\PY{p}{,}
             \PY{l+s+s1}{\PYZsq{}}\PY{l+s+s1}{v2}\PY{l+s+s1}{\PYZsq{}}\PY{p}{:} \PY{p}{[}\PY{l+s+s1}{\PYZsq{}}\PY{l+s+s1}{v4}\PY{l+s+s1}{\PYZsq{}}\PY{p}{,} \PY{l+s+s1}{\PYZsq{}}\PY{l+s+s1}{v5}\PY{l+s+s1}{\PYZsq{}}\PY{p}{]}\PY{p}{,}
             \PY{l+s+s1}{\PYZsq{}}\PY{l+s+s1}{v3}\PY{l+s+s1}{\PYZsq{}}\PY{p}{:} \PY{p}{[}\PY{l+s+s1}{\PYZsq{}}\PY{l+s+s1}{v6}\PY{l+s+s1}{\PYZsq{}}\PY{p}{]}\PY{p}{,}
             \PY{l+s+s1}{\PYZsq{}}\PY{l+s+s1}{v4}\PY{l+s+s1}{\PYZsq{}}\PY{p}{:} \PY{p}{[}\PY{l+s+s1}{\PYZsq{}}\PY{l+s+s1}{v3}\PY{l+s+s1}{\PYZsq{}}\PY{p}{,} \PY{l+s+s1}{\PYZsq{}}\PY{l+s+s1}{v6}\PY{l+s+s1}{\PYZsq{}}\PY{p}{,} \PY{l+s+s1}{\PYZsq{}}\PY{l+s+s1}{v7}\PY{l+s+s1}{\PYZsq{}}\PY{p}{]}\PY{p}{,}
             \PY{l+s+s1}{\PYZsq{}}\PY{l+s+s1}{v5}\PY{l+s+s1}{\PYZsq{}}\PY{p}{:} \PY{p}{[}\PY{l+s+s1}{\PYZsq{}}\PY{l+s+s1}{v4}\PY{l+s+s1}{\PYZsq{}}\PY{p}{,} \PY{l+s+s1}{\PYZsq{}}\PY{l+s+s1}{v7}\PY{l+s+s1}{\PYZsq{}}\PY{p}{]}\PY{p}{,}
             \PY{l+s+s1}{\PYZsq{}}\PY{l+s+s1}{v6}\PY{l+s+s1}{\PYZsq{}}\PY{p}{:} \PY{p}{[}\PY{p}{]}\PY{p}{,}
             \PY{l+s+s1}{\PYZsq{}}\PY{l+s+s1}{v7}\PY{l+s+s1}{\PYZsq{}}\PY{p}{:} \PY{p}{[}\PY{l+s+s1}{\PYZsq{}}\PY{l+s+s1}{v6}\PY{l+s+s1}{\PYZsq{}}\PY{p}{]}
         \PY{p}{\PYZcb{}}
\end{Verbatim}


    \begin{Verbatim}[commandchars=\\\{\}]
{\color{incolor}In [{\color{incolor}21}]:} \PY{n}{graph}
\end{Verbatim}


\begin{Verbatim}[commandchars=\\\{\}]
{\color{outcolor}Out[{\color{outcolor}21}]:} \{'v1': ['v2', 'v4', 'v3'],
          'v2': ['v4', 'v5'],
          'v3': ['v6'],
          'v4': ['v3', 'v6', 'v7'],
          'v5': ['v4', 'v7'],
          'v6': [],
          'v7': ['v6']\}
\end{Verbatim}
            
    \begin{itemize}
\tightlist
\item
  Find Path
\end{itemize}

    \begin{Verbatim}[commandchars=\\\{\}]
{\color{incolor}In [{\color{incolor}22}]:}  \PY{k}{def} \PY{n+nf}{find\PYZus{}path}\PY{p}{(}\PY{n}{graph}\PY{p}{,} \PY{n}{start}\PY{p}{,} \PY{n}{end}\PY{p}{,} \PY{n}{path}\PY{o}{=}\PY{p}{[}\PY{p}{]}\PY{p}{)}\PY{p}{:}
                 \PY{n}{path} \PY{o}{=} \PY{n}{path} \PY{o}{+} \PY{p}{[}\PY{n}{start}\PY{p}{]}
                 \PY{k}{if} \PY{n}{start} \PY{o}{==} \PY{n}{end}\PY{p}{:}
                     \PY{k}{return} \PY{n}{path}
                 \PY{k}{if} \PY{o+ow}{not} \PY{n}{start} \PY{o+ow}{in} \PY{n}{graph}\PY{o}{.}\PY{n}{keys}\PY{p}{(}\PY{p}{)}\PY{p}{:}
                     \PY{k}{return} \PY{k+kc}{None}
                 \PY{k}{for} \PY{n}{node} \PY{o+ow}{in} \PY{n}{graph}\PY{p}{[}\PY{n}{start}\PY{p}{]}\PY{p}{:}
                     \PY{k}{if} \PY{n}{node} \PY{o+ow}{not} \PY{o+ow}{in} \PY{n}{path}\PY{p}{:}
                         \PY{n}{newpath} \PY{o}{=} \PY{n}{find\PYZus{}path}\PY{p}{(}\PY{n}{graph}\PY{p}{,} \PY{n}{node}\PY{p}{,} \PY{n}{end}\PY{p}{,} \PY{n}{path}\PY{p}{)}
                         \PY{k}{if} \PY{n}{newpath}\PY{p}{:} \PY{k}{return} \PY{n}{newpath}
                 \PY{k}{return} \PY{k+kc}{None}
\end{Verbatim}


    \begin{Verbatim}[commandchars=\\\{\}]
{\color{incolor}In [{\color{incolor}23}]:} \PY{n}{find\PYZus{}path}\PY{p}{(}\PY{n}{graph}\PY{p}{,} \PY{l+s+s1}{\PYZsq{}}\PY{l+s+s1}{v1}\PY{l+s+s1}{\PYZsq{}}\PY{p}{,} \PY{l+s+s1}{\PYZsq{}}\PY{l+s+s1}{v6}\PY{l+s+s1}{\PYZsq{}}\PY{p}{)} \PY{c+c1}{\PYZsh{}clearly not the best path, just A path}
\end{Verbatim}


\begin{Verbatim}[commandchars=\\\{\}]
{\color{outcolor}Out[{\color{outcolor}23}]:} ['v1', 'v2', 'v4', 'v3', 'v6']
\end{Verbatim}
            
    I didnt really have time to do this in my own words but here is some
stuff I took from
\href{https://www.geeksforgeeks.org/breadth-first-search-or-bfs-for-a-graph/}{Geeks
for Geeks}

    \begin{Verbatim}[commandchars=\\\{\}]
{\color{incolor}In [{\color{incolor}24}]:} \PY{c+c1}{\PYZsh{} Python3 Program to print BFS traversal }
         \PY{c+c1}{\PYZsh{} from a given source vertex. BFS(int s) }
         \PY{c+c1}{\PYZsh{} traverses vertices reachable from s. }
         \PY{k+kn}{from} \PY{n+nn}{collections} \PY{k}{import} \PY{n}{defaultdict} 
           
         \PY{c+c1}{\PYZsh{} This class represents a directed graph }
         \PY{c+c1}{\PYZsh{} using adjacency list representation }
         \PY{k}{class} \PY{n+nc}{Graph}\PY{p}{:} 
           
             \PY{c+c1}{\PYZsh{} Constructor }
             \PY{k}{def} \PY{n+nf}{\PYZus{}\PYZus{}init\PYZus{}\PYZus{}}\PY{p}{(}\PY{n+nb+bp}{self}\PY{p}{)}\PY{p}{:} 
           
                 \PY{c+c1}{\PYZsh{} default dictionary to store graph }
                 \PY{n+nb+bp}{self}\PY{o}{.}\PY{n}{graph} \PY{o}{=} \PY{n}{defaultdict}\PY{p}{(}\PY{n+nb}{list}\PY{p}{)} 
           
             \PY{c+c1}{\PYZsh{} function to add an edge to graph }
             \PY{k}{def} \PY{n+nf}{addEdge}\PY{p}{(}\PY{n+nb+bp}{self}\PY{p}{,}\PY{n}{u}\PY{p}{,}\PY{n}{v}\PY{p}{)}\PY{p}{:} 
                 \PY{n+nb+bp}{self}\PY{o}{.}\PY{n}{graph}\PY{p}{[}\PY{n}{u}\PY{p}{]}\PY{o}{.}\PY{n}{append}\PY{p}{(}\PY{n}{v}\PY{p}{)} 
           
             \PY{c+c1}{\PYZsh{} Function to print a BFS of graph }
             \PY{k}{def} \PY{n+nf}{BFS}\PY{p}{(}\PY{n+nb+bp}{self}\PY{p}{,} \PY{n}{s}\PY{p}{)}\PY{p}{:} 
           
                 \PY{c+c1}{\PYZsh{} Mark all the vertices as not visited }
                 \PY{n}{visited} \PY{o}{=} \PY{p}{[}\PY{k+kc}{False}\PY{p}{]} \PY{o}{*} \PY{p}{(}\PY{n+nb}{len}\PY{p}{(}\PY{n+nb+bp}{self}\PY{o}{.}\PY{n}{graph}\PY{p}{)}\PY{p}{)} 
           
                 \PY{c+c1}{\PYZsh{} Create a queue for BFS }
                 \PY{n}{queue} \PY{o}{=} \PY{p}{[}\PY{p}{]} 
           
                 \PY{c+c1}{\PYZsh{} Mark the source node as  }
                 \PY{c+c1}{\PYZsh{} visited and enqueue it }
                 \PY{n}{queue}\PY{o}{.}\PY{n}{append}\PY{p}{(}\PY{n}{s}\PY{p}{)} 
                 \PY{n}{visited}\PY{p}{[}\PY{n}{s}\PY{p}{]} \PY{o}{=} \PY{k+kc}{True}
           
                 \PY{k}{while} \PY{n}{queue}\PY{p}{:} 
           
                     \PY{c+c1}{\PYZsh{} Dequeue a vertex from  }
                     \PY{c+c1}{\PYZsh{} queue and print it }
                     \PY{n}{s} \PY{o}{=} \PY{n}{queue}\PY{o}{.}\PY{n}{pop}\PY{p}{(}\PY{l+m+mi}{0}\PY{p}{)} 
                     \PY{n+nb}{print} \PY{p}{(}\PY{n}{s}\PY{p}{,} \PY{n}{end} \PY{o}{=} \PY{l+s+s2}{\PYZdq{}}\PY{l+s+s2}{ }\PY{l+s+s2}{\PYZdq{}}\PY{p}{)} 
           
                     \PY{c+c1}{\PYZsh{} Get all adjacent vertices of the }
                     \PY{c+c1}{\PYZsh{} dequeued vertex s. If a adjacent }
                     \PY{c+c1}{\PYZsh{} has not been visited, then mark it }
                     \PY{c+c1}{\PYZsh{} visited and enqueue it }
                     \PY{k}{for} \PY{n}{i} \PY{o+ow}{in} \PY{n+nb+bp}{self}\PY{o}{.}\PY{n}{graph}\PY{p}{[}\PY{n}{s}\PY{p}{]}\PY{p}{:} 
                         \PY{k}{if} \PY{n}{visited}\PY{p}{[}\PY{n}{i}\PY{p}{]} \PY{o}{==} \PY{k+kc}{False}\PY{p}{:} 
                             \PY{n}{queue}\PY{o}{.}\PY{n}{append}\PY{p}{(}\PY{n}{i}\PY{p}{)} 
                             \PY{n}{visited}\PY{p}{[}\PY{n}{i}\PY{p}{]} \PY{o}{=} \PY{k+kc}{True}
           
         \PY{c+c1}{\PYZsh{} Driver code }
           
         \PY{c+c1}{\PYZsh{} Create a graph given in }
         \PY{c+c1}{\PYZsh{} the above diagram }
         \PY{n}{g} \PY{o}{=} \PY{n}{Graph}\PY{p}{(}\PY{p}{)} 
         \PY{n}{g}\PY{o}{.}\PY{n}{addEdge}\PY{p}{(}\PY{l+m+mi}{0}\PY{p}{,} \PY{l+m+mi}{1}\PY{p}{)} 
         \PY{n}{g}\PY{o}{.}\PY{n}{addEdge}\PY{p}{(}\PY{l+m+mi}{0}\PY{p}{,} \PY{l+m+mi}{2}\PY{p}{)} 
         \PY{n}{g}\PY{o}{.}\PY{n}{addEdge}\PY{p}{(}\PY{l+m+mi}{1}\PY{p}{,} \PY{l+m+mi}{2}\PY{p}{)} 
         \PY{n}{g}\PY{o}{.}\PY{n}{addEdge}\PY{p}{(}\PY{l+m+mi}{2}\PY{p}{,} \PY{l+m+mi}{0}\PY{p}{)} 
         \PY{n}{g}\PY{o}{.}\PY{n}{addEdge}\PY{p}{(}\PY{l+m+mi}{2}\PY{p}{,} \PY{l+m+mi}{3}\PY{p}{)} 
         \PY{n}{g}\PY{o}{.}\PY{n}{addEdge}\PY{p}{(}\PY{l+m+mi}{3}\PY{p}{,} \PY{l+m+mi}{3}\PY{p}{)} 
           
         \PY{n+nb}{print} \PY{p}{(}\PY{l+s+s2}{\PYZdq{}}\PY{l+s+s2}{Following is Breadth First Traversal}\PY{l+s+s2}{\PYZdq{}}
                           \PY{l+s+s2}{\PYZdq{}}\PY{l+s+s2}{ (starting from vertex 2)}\PY{l+s+s2}{\PYZdq{}}\PY{p}{)} 
         \PY{n}{g}\PY{o}{.}\PY{n}{BFS}\PY{p}{(}\PY{l+m+mi}{2}\PY{p}{)} 
           
         \PY{c+c1}{\PYZsh{} This code is contributed by Neel}
\end{Verbatim}


    \begin{Verbatim}[commandchars=\\\{\}]
Following is Breadth First Traversal (starting from vertex 2)
2 0 3 1 
    \end{Verbatim}

    aaaand for
\href{https://www.geeksforgeeks.org/depth-first-search-or-dfs-for-a-graph/}{DFS}

    \begin{Verbatim}[commandchars=\\\{\}]
{\color{incolor}In [{\color{incolor}25}]:} \PY{c+c1}{\PYZsh{} Python program to print DFS traversal from a }
         \PY{c+c1}{\PYZsh{} given given graph }
         \PY{k+kn}{from} \PY{n+nn}{collections} \PY{k}{import} \PY{n}{defaultdict} 
           
         \PY{c+c1}{\PYZsh{} This class represents a directed graph using }
         \PY{c+c1}{\PYZsh{} adjacency list representation }
         \PY{k}{class} \PY{n+nc}{Graph}\PY{p}{:} 
           
             \PY{c+c1}{\PYZsh{} Constructor }
             \PY{k}{def} \PY{n+nf}{\PYZus{}\PYZus{}init\PYZus{}\PYZus{}}\PY{p}{(}\PY{n+nb+bp}{self}\PY{p}{)}\PY{p}{:} 
           
                 \PY{c+c1}{\PYZsh{} default dictionary to store graph }
                 \PY{n+nb+bp}{self}\PY{o}{.}\PY{n}{graph} \PY{o}{=} \PY{n}{defaultdict}\PY{p}{(}\PY{n+nb}{list}\PY{p}{)} 
           
             \PY{c+c1}{\PYZsh{} function to add an edge to graph }
             \PY{k}{def} \PY{n+nf}{addEdge}\PY{p}{(}\PY{n+nb+bp}{self}\PY{p}{,}\PY{n}{u}\PY{p}{,}\PY{n}{v}\PY{p}{)}\PY{p}{:} 
                 \PY{n+nb+bp}{self}\PY{o}{.}\PY{n}{graph}\PY{p}{[}\PY{n}{u}\PY{p}{]}\PY{o}{.}\PY{n}{append}\PY{p}{(}\PY{n}{v}\PY{p}{)} 
           
             \PY{c+c1}{\PYZsh{} A function used by DFS }
             \PY{k}{def} \PY{n+nf}{DFSUtil}\PY{p}{(}\PY{n+nb+bp}{self}\PY{p}{,}\PY{n}{v}\PY{p}{,}\PY{n}{visited}\PY{p}{)}\PY{p}{:} 
           
                 \PY{c+c1}{\PYZsh{} Mark the current node as visited and print it }
                 \PY{n}{visited}\PY{p}{[}\PY{n}{v}\PY{p}{]}\PY{o}{=} \PY{k+kc}{True}
                 \PY{n+nb}{print}\PY{p}{(}\PY{n}{v}\PY{p}{)} 
           
                 \PY{c+c1}{\PYZsh{} Recur for all the vertices adjacent to this vertex }
                 \PY{k}{for} \PY{n}{i} \PY{o+ow}{in} \PY{n+nb+bp}{self}\PY{o}{.}\PY{n}{graph}\PY{p}{[}\PY{n}{v}\PY{p}{]}\PY{p}{:} 
                     \PY{k}{if} \PY{n}{visited}\PY{p}{[}\PY{n}{i}\PY{p}{]} \PY{o}{==} \PY{k+kc}{False}\PY{p}{:} 
                         \PY{n+nb+bp}{self}\PY{o}{.}\PY{n}{DFSUtil}\PY{p}{(}\PY{n}{i}\PY{p}{,} \PY{n}{visited}\PY{p}{)} 
           
           
             \PY{c+c1}{\PYZsh{} The function to do DFS traversal. It uses }
             \PY{c+c1}{\PYZsh{} recursive DFSUtil() }
             \PY{k}{def} \PY{n+nf}{DFS}\PY{p}{(}\PY{n+nb+bp}{self}\PY{p}{,}\PY{n}{v}\PY{p}{)}\PY{p}{:} 
           
                 \PY{c+c1}{\PYZsh{} Mark all the vertices as not visited }
                 \PY{n}{visited} \PY{o}{=} \PY{p}{[}\PY{k+kc}{False}\PY{p}{]}\PY{o}{*}\PY{p}{(}\PY{n+nb}{len}\PY{p}{(}\PY{n+nb+bp}{self}\PY{o}{.}\PY{n}{graph}\PY{p}{)}\PY{p}{)} 
           
                 \PY{c+c1}{\PYZsh{} Call the recursive helper function to print }
                 \PY{c+c1}{\PYZsh{} DFS traversal }
                 \PY{n+nb+bp}{self}\PY{o}{.}\PY{n}{DFSUtil}\PY{p}{(}\PY{n}{v}\PY{p}{,}\PY{n}{visited}\PY{p}{)} 
           
           
         \PY{c+c1}{\PYZsh{} Driver code }
         \PY{c+c1}{\PYZsh{} Create a graph given in the above diagram }
         \PY{n}{g} \PY{o}{=} \PY{n}{Graph}\PY{p}{(}\PY{p}{)} 
         \PY{n}{g}\PY{o}{.}\PY{n}{addEdge}\PY{p}{(}\PY{l+m+mi}{0}\PY{p}{,} \PY{l+m+mi}{1}\PY{p}{)} 
         \PY{n}{g}\PY{o}{.}\PY{n}{addEdge}\PY{p}{(}\PY{l+m+mi}{0}\PY{p}{,} \PY{l+m+mi}{2}\PY{p}{)} 
         \PY{n}{g}\PY{o}{.}\PY{n}{addEdge}\PY{p}{(}\PY{l+m+mi}{1}\PY{p}{,} \PY{l+m+mi}{2}\PY{p}{)} 
         \PY{n}{g}\PY{o}{.}\PY{n}{addEdge}\PY{p}{(}\PY{l+m+mi}{2}\PY{p}{,} \PY{l+m+mi}{0}\PY{p}{)} 
         \PY{n}{g}\PY{o}{.}\PY{n}{addEdge}\PY{p}{(}\PY{l+m+mi}{2}\PY{p}{,} \PY{l+m+mi}{3}\PY{p}{)} 
         \PY{n}{g}\PY{o}{.}\PY{n}{addEdge}\PY{p}{(}\PY{l+m+mi}{3}\PY{p}{,} \PY{l+m+mi}{3}\PY{p}{)} 
           
         \PY{n+nb}{print}\PY{p}{(}\PY{l+s+s2}{\PYZdq{}}\PY{l+s+s2}{Following is DFS from (starting from vertex 2)}\PY{l+s+s2}{\PYZdq{}}\PY{p}{)}
         \PY{n}{g}\PY{o}{.}\PY{n}{DFS}\PY{p}{(}\PY{l+m+mi}{2}\PY{p}{)} 
           
         \PY{c+c1}{\PYZsh{} This code is contributed by Neelam Yadav }
\end{Verbatim}


    \begin{Verbatim}[commandchars=\\\{\}]
Following is DFS from (starting from vertex 2)
2
0
1
3

    \end{Verbatim}


    % Add a bibliography block to the postdoc
    
    
    
    \end{document}
